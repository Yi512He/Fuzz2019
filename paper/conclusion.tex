
Although there are different critiques against fuzz testing, the effectiveness of fuzz testing is still sharp. After over three decades since the first fuzz study, the overall crash/hang rate of common utilities on three Unix variant platform is 24\%. We believe with more pushes on existing fuzz tool, systematically randomizing command line arguments, and expanding our tools on X-windows, more crashes/hangs would be uncovered. 

Based on analyzing the above results from fuzz testing, we have made three observations in comparison with previous studies. The versions of utilities matter, because bugs are more version specific, such as \texttt{flex} and \texttt{find}. Based on the versions of utilities we studied, some bugs have been fixed, such as end-of-file and divide by zero. Though other bugs are still prevalent, such as pointer/array, they become less vulnerable and were embedded in the deeper layer. In most cases, they do not directly crash or hang utilities but are responsible for later failing behaviors. 

In conclusion, our data shows that the number of failing utilities on Linux has largely increased since 1995, whereas FreeBSD and MacOS have a lower failing rate than Linux. Xv6 could be more robust for educational purposes. Some of the bugs detected in previous studies still persist and additional bugs have emerged with the development of systems. Hence, we believe that more platforms and utilities will benefit from fuzz testing tools.